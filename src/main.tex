% CV in LaTeX — Gustavo Rohrer
% Engine: XeLaTeX (MacTeX)
\documentclass[10pt,a4paper]{article}

% Typography & layout
\usepackage[a4paper,top=0.45in,bottom=0.6in,left=0.5in,right=0.5in]{geometry}
\usepackage{fontspec}
\usepackage{titlesec}
\usepackage{titling}
% Pull the title block up a bit
\setlength{\droptitle}{-1.2em}
\usepackage{enumitem}
\usepackage{tabularx}
\usepackage{hyperref}
\usepackage{fancyhdr}
\usepackage{datetime2}
\usepackage{xcolor}

% Fonts (prefer Georgia on macOS, fallback to TeX Gyre Pagella)
\newif\ifgeorgia
\IfFontExistsTF{Georgia}{\georgiatrue}{\georgiafalse}
\ifgeorgia
  \setmainfont{Georgia}
\else
  \setmainfont{TeX Gyre Pagella}
\fi

% Section style — small caps, rule underneath
\titleformat{\section}{\large\bfseries}{}{0pt}{\MakeUppercase}[
  \vspace{0.2em}\titlerule]
\titlespacing*{\section}{0pt}{0.45em}{0.2em}

% Reduce space between items and paragraphs
\setlength{\parskip}{0.35em}
\setlength{\parindent}{0pt}
\setlist[itemize]{leftmargin=1em,labelsep=0.35em,itemsep=0.16em,topsep=0.12em,parsep=0pt,partopsep=0pt}

% Colors
\definecolor{muted}{gray}{0.3}
\hypersetup{
  colorlinks=true,
  urlcolor=black,
  linkcolor=black
}

% Header (title block)
\pretitle{\begin{center}\LARGE\bfseries}
\posttitle{\par\end{center}\vspace{-1.4em}}
\preauthor{\begin{center}\normalsize}
\postauthor{\par\end{center}\vspace{-0.8em}}

% Footer with last updated date
\pagestyle{fancy}
\fancyhf{}
\lfoot{\footnotesize Last updated: \DTMtoday}
\cfoot{\footnotesize Gustavo Rohrer}
\rfoot{\thepage}
\renewcommand{\headrulewidth}{0pt}
\renewcommand{\footrulewidth}{0pt}

% Ensure the same footer also appears on pages using the 'plain' style (e.g., title page)
\fancypagestyle{plain}{
  \fancyhf{}
  \lfoot{\footnotesize Last updated: \DTMtoday}
  \cfoot{\footnotesize Gustavo Rohrer}
  \rfoot{\thepage}
  \renewcommand{\headrulewidth}{0pt}
  \renewcommand{\footrulewidth}{0pt}
}

% Helper: two-column row with flexible left and right
\newcommand{\tworow}[2]{\noindent\begin{tabularx}{\textwidth}{@{}X r@{}}#1 & #2\\\end{tabularx}}

% Helper: role/company/date/location header
\newcommand{\roleentry}[4]{%
  % #1 Title, #2 Company, #3 Right column (e.g., Remote), #4 Dates
  \textbf{#1} \hfill {\small #3}\\[-0.2em]
  {\itshape #2} \hfill {\small #4}\\[-0.35em]
}

\begin{document}

% ---------- Title ----------
\title{Gustavo Rohrer}
\author{Argentina — Remote-first}
\date{\vspace{-1.1em}\small \href{https://www.linkedin.com/in/gustavo-rohrer-89b00443}{LinkedIn} \;|\; \href{mailto:gustavorohrer@gmail.com}{gustavorohrer@gmail.com}}
\maketitle

% ---------- About Me ----------
\section*{About Me}
Full-stack Engineer with over 10 years of experience. Expert in building scalable systems with **Go (3+ years)** and **React/Next.js (7+ years)**, with a recent focus on **Effect-TS** for type-safe, functional backend architectures.

% ---------- Experience ----------
\section*{Experience}

\roleentry{Product Engineer}{ELYS Health}{Remote}{2026\textendash{}2026}
\begin{itemize}
  \item Subscription and Payment Infrastructure: Architected a comprehensive subscription system using Stripe, managing the full lifecycle from Checkout to Webhook handling; ensured consistent membership state through an event-driven architecture and oRPC transport.
  \item Onboarding UX Redesign: Re-engineered the onboarding flow by migrating from modal-based state to Next.js named routes, enabling native browser history support (Back/Forward) and robust state persistence, which streamlined the user registration experience and simplified maintenance.
  \item Critical Incident Remediation: Led the resolution of a high-priority Clerk webhook infinite loop, automating the cleanup of 120k+ corrupted records and implementing regression tests that stabilized the authentication layer.
  \item Scalable User Growth System: Developed an automated waitlist and invitation system integrated with Resend, using atomic processing patterns to ensure data integrity and reliable lead management.
  \item Architecture and Standards: Established core project guidelines and utility organization within a monorepo, adopting Effect-TS for type-safe functional programming, impacting the development workflow of a team of four engineers.
\end{itemize}

\vspace{0.3em}
\roleentry{Full-stack Engineer}{Epwery}{Remote}{2022\textendash{}2025}
\begin{itemize}
  \item Designed and implemented Go backend services integrated with Salesforce and Stripe.
  \item Built an idempotent Stripe webhook reconciliation worker, reducing payment mismatches and improving collection accuracy.
  \item Developed automated email notification workflows, improving reliability and user engagement.
  \item Increased test coverage on critical modules, reducing post-release bugs by up to 50\%.
  \item Optimized CI pipelines with selective tests, saving 5\textendash{}15 minutes per PR and improving delivery speed.
  \item Contributed to React components (including portals) and performed daily cross-team code reviews.
\end{itemize}

\vspace{0.3em}
\roleentry{Full-stack Engineer \textemdash{} Prior roles}{Various companies}{Remote \& On-site}{2015\textendash{}2022}
\begin{itemize}
  \item Delivered production features in Java, Angular, and Node.js.
  \item Participated in sprint planning and code reviews, and coordinated tasks across different teams.
  \item DropCar: Migrated a monolithic architecture to microservices and integrated Twilio for masked communications.
  \item COVID\textendash{}19 project: Led frontend development, acted as Scrum Master, and improved UX under tight deadlines.
  \item Synegen: Designed badge assignment logic for an e-learning platform (backend and UI).
\end{itemize}

% ---------- Skills ----------
\section*{Skills}
Go, React, Next.js, TypeScript, Effect-TS, Drizzle ORM, PostgreSQL, AWS, Java, Node.js, Python, oRPC, Stripe, Clerk, Resend, REST APIs, GraphQL, Docker, CI/CD, Testing (Vitest/mocks), Microservices, Observability and Logging.

% ---------- Education ----------
\section*{Education}
\tworow{\textbf{Universidad Nacional de La Plata, Argentina}}{\textit{Completed}}\\[-0.2em]
\textit{Bachelor's degree in Information Systems}

\end{document}
